\chapter{Revisão Bibliográfica}

A área de modelagem, identificação e controle de sistemas tem uma ampla gama de técnicas matemáticas que auxiliam os engenheiros a abstrair sistemas físicos, e com isso, generalizar modelos para que possam ser simulados computacionalmente, além disso, a partir dos modelos encontrados é possível projetar sistemas de controle precisos e complexos com esse mesmo conjuntos de ferramentas matemáticas.

As técnicas matemáticas que são amplamente usada quando se trata de sistemas de controle são, Função de Transferência no domínio da frequência, Espaço de Estados e modelos discretizados usado transformada Z.

além disso, sistemas computacionais são necessários tanto para simular, projetar ou mesmo implementar sistemas de controle, dessa forma, a eletrônica analógica e digital é parte fundamental para o engenheiro de controle fazendo a comunicação do mundo analógico com o digital.

para esse capítulo, será feita a revisão bibliográfica dos tópicos de controle e modelagem de sistema usados 
nessa monografia. 

\section{Transformada de Laplace}

\section{Função de Transferência}

\section{Espaço de Estados}

\section{Transformada Z}

