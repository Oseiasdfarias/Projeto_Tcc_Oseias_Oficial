% CAPA-------------------------------------------------------------------------------------

% ORIENTAÇÕES GERAIS-------------------------------------------------------------------------------------
% Caso algum dos campos não se aplique ao seu trabalho, como por exemplo,
% se não houve coorientador, apenas deixe vazio.
% Exemplos: 
% \coorientador{}
% \departamento{}

% DADOS DO TRABALHO------------------------------------------------------------------------
\titulo{\textbf{TÍTULO DO TRABALHO}}
\subtitulo{subtítulo (se houver)} %Se não hover subtítulo, deixar em branco.
\titleabstract{Title in English}
\autor{Oséias Dias de Farias}
\autorcitacao{ÚLTIMO NOME, Nome} % Sobrenome em maiúsculo
\local{TUCURUÍ/PA}
\data{2023}

% NATUREZA DO TRABALHO-----------------------------------------------------------------------------------
\projeto{Trabalho de Conclusão de Curso}

% TÍTULO ACADÊMICO-------------------------------------------------------------------------
\tituloAcademico{Bacharel}

% ÁREA DE CONCENTRAÇÃO E LINHA DE PESQUISA-----------------------------------------------------------------------------------
% Se a natureza for Trabalho de Conclusão de Curso, deixe ambos os campos vazios
% Se for programa de Pós-graduação, indique a área de concentração e a linha de pesquisa
\areaconcentracao{}
\linhapesquisa{}

% DADOS DA INSTITUIÇÃO---------------------------------------------------------------------
% Se a natureza for Trabalho de Conclusão de Curso, coloque o nome do curso de graduação em "programa"
% Formato para o logo da Instituição: \logoinstituicao{<escala>}{<caminho/nome do arquivo>}
\instituicao{Universidade Federal do Pará}
\departamento{Instituto de Tecnologia}
\programa{Faculdade de Engenharia Elétrica}
\logoinstituicao{2cm}{figuras/naomexafig/logoufpa.png} %

% DADOS DOS ORIENTADORES-------------------------------------------------------------------
\orientador{Prof. Esp. XXXX XXXXX XXXXXXX}
%\orientador[Orientadora:]{Nome da orientadora}
\instOrientador{Universidade Federal do Pará}

%\coorientador{Nome do coorientador}
%\coorientador[Coorientadora:]{Nome da coorientadora}
%\instCoorientador{Instituição do coorientador}
