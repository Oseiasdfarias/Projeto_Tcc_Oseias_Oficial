% Lista de abreviações
\makeatletter
\newcommand\criarsimbolo[2]{%
	\write\@auxout{\noexpand\@writefile{sbl}{\noexpand\item[#1] #2}}}
\newcommand\imprimirlistadesimbolos{%
	\begin{simbolos}
		\@starttoc{sbl}
	\end{simbolos}}
\makeatother



% Estilo ICMC
\makechapterstyle{icmc}{%
  \renewcommand{\chapterheadstart}{}


%  ##########  MINHAS MODIFICAÇÕES  ##########

    % Secao secundaria (Section) Caixa baixa, Negrito
    %\renewcommand*{\cftsectionfont}{\bfseries}
    % Secao terciaria (Subsection) Caixa baixa, Negrito, italico
    %\renewcommand*{\cftsubsectionfont}{\itshape\bfseries}
    % Secao quaternaria (Subsubsection) Caixa baixa, italico
    %\renewcommand*{\cftsubsubsectionfont}{\itshape}
    % Secao quinquenária (Subsubsubsection) Caixa baixa
    %\renewcommand*{\cftparagraphfont}{\normalsize}


  % tamanhos de fontes de chapter e part
   \ifthenelse{\equal{\ABNTEXisarticle}{true}}{%
     \setlength\beforechapskip{\baselineskip}
     \renewcommand{\chaptitlefont}{\ABNTEXsectionfont\ABNTEXsectionfontsize}
   }{%else
     \setlength{\beforechapskip}{0pt}
     \renewcommand{\ABNTEXchapterfontsize}{\LARGE}

    %\renewcommand{\ABNTEXchapter}{\sffamily\bfseries}  %alteração da fonte dos capítulos, seções e subseções
    % \renewcommand{\chaptitlefont}{\ABNTEXchapterfont\bfseries\ABNTEXchapterfontsize}
   }

  \renewcommand{\chapnumfont}{\chaptitlefont}
  \renewcommand{\parttitlefont}{\ABNTEXpartfont\ABNTEXpartfontsize}
  \renewcommand{\partnumfont}{\ABNTEXpartfont\ABNTEXpartfontsize}
  \renewcommand{\partnamefont}{\ABNTEXpartfont\ABNTEXpartfontsize}
  
  % Modifica a Fonte das sessões etc para time new roma;
  %\renewcommand{\ABNTEXsectionfont}{\fontfamily{ptm}\selectfont\bfseries}
  

  % tamanhos de fontes de section, subsection, subsubsection e subsubsubsection
  \setsecheadstyle{\ABNTEXsectionfont\ABNTEXsectionfontsize\bfseries\ABNTEXsectionupperifneeded}
  \setsubsecheadstyle{\ABNTEXsubsectionfont\ABNTEXsubsectionfontsize\bfseries\ABNTEXsubsectionupperifneeded}
  \setsubsubsecheadstyle{\ABNTEXsubsubsectionfont\ABNTEXsubsubsectionfontsize\ABNTEXsubsubsectionupperifneeded}
  % \setsubsecheadstyle{\ABNTEXsubsectionfont\ABNTEXsubsectionfontsize\bfseries\itshape\ABNTEXsubsectionupperifneeded}
  % \setsubsubsecheadstyle{\ABNTEXsubsubsectionfont\ABNTEXsubsubsectionfontsize\itshape\ABNTEXsubsubsectionupperifneeded}
  \setsubsubsubsecheadstyle{\ABNTEXsubsubsubsectionfont\ABNTEXsubsubsubsectionfontsize\ABNTEXsubsubsubsectionupperifneeded}

  % impressao do numero do capitulo
  \renewcommand{\chapternamenum}{}

  % impressao do nome do capitulo
  \renewcommand{\printchaptername}{%
   %\chaptitlefont
   %\ifthenelse{\boolean{abntex@apendiceousecao}}{\appendixname}{\chaptername}%
  }

  % impressao do titulo do capitulo
  \def\printchaptertitle##1{%

    \setboolean{ABNTEXupperchapter}{true}

    \ifthenelse{\boolean{abntex@innonumchapter}}{
        \vskip 0ex \hrulefill\chaptitlefont\bfseries\ABNTEXchapterupperifneeded{##1}
        \vskip -0.6ex\hfill\rule{.8\textwidth}{0.5pt}
        \vskip -2.8ex\hfill\rule{.8\textwidth}{2pt}
        \vskip 1.5ex

    }{%
    % else
        {\hrulefill
        {\renewcommand{\arraystretch}{1.5} %  1 is the default, change whatever you need
        \begin{tabular}{|c|}
            \rowcolor{black}\color{white}\normalsize\ABNTEXchapterfont
              \ifthenelse{\boolean{abntex@apendiceousecao}}{\MakeTextUppercase{\appendixname}}{\MakeTextUppercase{\chaptername}}  \\
            \vspace{-1.5ex}\\ %coloquei para aumentar o espaço entre o título e o número
            \resizebox{!}{1.1cm}{\ABNTEXchapterfont\thechapter}
            \\[2.5ex]
            \hline
        \end{tabular}}} \\
        \vskip 4.3ex \flushright\chaptitlefont\bfseries\ABNTEXchapterupperifneeded{##1} \\
        \vskip -0.6ex\hfill\rule{.8\textwidth}{0.5pt} \\
        \vskip -2.8ex\hfill\rule{.8\textwidth}{2pt}\\
        \vskip 1.5ex
	}

  }

  % impressao do numero do capitulo
  \renewcommand{\printchapternum}{%
    \setboolean{abntex@innonumchapter}{false}
  }
  \renewcommand{\afterchapternum}{}

  % impressao do capitulo nao numerado
  \renewcommand\printchapternonum{%
     \setboolean{abntex@innonumchapter}{true}%
    }
}
\chapterstyle{icmc}

% Modifica a fonte para Times New Roma
%\usepackage{newtxtext}
%\usepackage{newtxmath}


