% INTRODUÇÃO------------------------------------------------------

\chapter{INTRODUÇÃO}\label{cap1}

Recentemente, tem-se notado um aumento significativo no número de aplicações que demandam cada vez mais processamento de dados, tais como mapeamento genético, computação gráfica, previsões meteorológicas e programas com grande quantidade de variáveis de entrada. \citeonline{dowd2010high} descrevem que a computação de alto desempenho é uma área que se preocupa em criar condições para executar processamentos com carga computacional elevada em intervalos de tempo viáveis.

Quando uma aplicação exige mais capacidade de processamento do que os computadores convencionais podem oferecer, é necessário buscar alternativas para criar sistemas de alto desempenho. Essa busca por soluções de processamento de alta performance é essencial para possibilitar a execução de programas complexos em um intervalo de tempo aceitável, e vem impulsionando o desenvolvimento de novas tecnologias de hardware e software. O uso de técnicas de processamento paralelo, computação distribuída e processamento em nuvem são exemplos de abordagens que visam a criação de sistemas computacionais de alto desempenho \cite{dowd2010high}. Essas tecnologias permitem que programas que levariam anos para fornecerem resultados em máquinas comuns possam ser executados em muito menos tempo, em questão de horas ou dias.

Uma abordagem diferente é a que consiste em construir circuitos orientados a aplicação, que são capazes de fornecer um bom desempenho para essa aplicação específica, ou seja, construir um hardware dedicado a essa aplicação. Por exemplo, é possível projetar e fabricar um circuito integrado específico (ASIC do inglês, \textit{Application Specific Integrated Circuit}) com o controle e as unidades funcionais personalizadas e otimizadas para uma dada aplicação. Com essa técnica de hardware dedicado pode-se atingir resultados muito bons com menos recursos de hardware \cite{Hager2017}. O ASIC é um circuito integrado para uma aplicação específica, tendo por objetivo solucionar demandas do mercado. Entretanto, cada projeto exige alto investimento, requerendo meses ou anos para ser desenvolvido, e fica limitado a uma dada aplicação. O desenvolvimento de ASICs é justificado comercialmente com um alto volume de produção, ou aplicações muito específicas, como em medicina (por exemplo, marcapasso) ou em aplicações militares.

Outra opção de dispositivo reconfigurável, é o FPGA (do inglês \textit{Field-programmable Gate Array}). Este é um dispositivo que permite a implementação de aplicações específicas. E contrastando com o ASICs, a implementação em FPGAs não é permanente, podendo ser reconfigurada de forma barata e fácil. Como vantagem, este apresenta baixa granularidade, permitindo a configuração em nível de porta lógica. Isso significa que é possível criar virtualmente qualquer circuito digital a partir de uma descrição em HDL (do inglês, \textit{Hardware Description Language}). 

Assim como um computador, o FPGA é capaz de executar milhões de operações simultâneas, tornando-o potencialmente centenas de vezes mais rápido do que projetos baseados em microprocessadores, devido à possibilidade de explorar o paralelismo espacial oferecido pelos elementos configuráveis presentes nos dispositivos atuais. 

FPGAs podem ser utilizados em diversas aplicações de sistemas embarcados, como no setor automotivo, médico, industrial, equipamentos de imagem e telecomunicações, por exemplo. Compete ao meio acadêmico e a indústria realizar estudos e incentivos para utilizar a combinação: lógica programável com processadores embarcados. Dado o crescente destaque deste campo de atuação, é interessante que conceitos de programação reconfigurável sejam introduzidos ao ensino de automação e eletrônica. Desta forma, este trabalho busca contribuir com o ensino de Engenharia, utilizando o kit FPGA DE2-115 da Altera, por meio de uma abordagem pratica obtida pela criação e execução de roteiros aplicáveis em sala de aula. O DE2-115 é uma placa de desenvolvimento que incorpora uma FPGA da Altera, juntamente com uma variedade de periféricos e interfaces, tornando-o ideal para projetos de automação. Isso permite que os estudantes aprendam conceitos importantes, como programação em VHDL (VHSIC\textit{ Hardware Description Language}), design de sistemas digitais, interfaceamento com sensores e atuadores, e implementação de lógica de controle em hardware. Os estudantes podem projetar e implementar sistemas de automação em tempo real, utilizando a plataforma FPGA para desenvolver circuitos personalizados e reconfiguráveis.

Através do ensino de automação utilizando o DE2-115, os estudantes podem adquirir habilidades práticas e relevantes para a indústria, como design de sistemas embarcados, programação de dispositivos reconfiguráveis, desenvolvimento de algoritmos de controle em hardware e integração de sistemas de automação com o mundo físico. Além disso, o uso do DE2-115 pode incentivar a criatividade e a inovação, permitindo que os estudantes projetem soluções personalizadas para problemas específicos de automação.


\section {Justificativa}

O uso de hardware reconfigurável permite a operação dedicada a um objetivo específico, sem influência de outros processos em concorrência, o que resulta em ganho de desempenho e processamento dedicado para a execução de algoritmos. É essencial que os profissionais de engenharia elétrica sejam introduzidos a ferramentas de programação digital, especialmente para análise e aplicação em campos como telecomunicações. No entanto, muitos estudantes enfrentam dificuldades nesses temas de estudo devido à escassez de materiais de ensino disponíveis. Portanto, esse trabalho propõe o desenvolvimento de material de ensino da linguagem VHDL, visando sua implementação em componentes curriculares de eletrônica digital e telecomunicações. 
 

\section{Objetivos}

Este trabalho tem como objetivo, propor a criação de material didático, e de fácil compreensão, para ensino de automação a partir de Dispositivos Lógicos Programáveis, utilizando um Kit de Aprendizado e Desenvolvimento FPGA (DE2-115), programado via VHDL.

Para alcançar o objetivo proposto, foram definidas seguintes metas:

\begin{enumerate}[label={\noNalph{enumi})}]
    \item Realizar uma revisão bibliográfica sobre os fundamentos de sistemas digitais e lógica de programação;
    \item Estudar a implementação de códigos em VHDL;
    \item Estudar o funcionamento do FPGA DE2-115 da Altera;
    \item Propor roteiros experimentais para estudo de lógica combinacional e sequencial;
    \item Documentar os resultados obtidos, contribuindo com o ensino de sistemas digitais e eletrônica reconfigurável.”
\end{enumerate}

\section{Estrutura da Monografia}
Este trabalho está estruturado em seis capítulos.

O capítulo \ref{cap1} é introdutório, neste são apresentados a justificativa, objetivos e estrutura do trabalho desenvolvido.

No capítulo 2 é apresentada a fundamentação teórica sobre sistemas digitais, lógica programável e dispositivos lógicos programáveis.

O capítulo 3 reúne as principais partes e funcionamento do kit FPGA DE2-115 da Altera.

No capítulo 4 são apresentados os roteiros experimentais que poderão ser aplicados para ensino de automação utilizando sistemas lógicos programáveis.

O capítulo 5 apresenta os resultados obtidos a partir da execução dos experimentos descritos no capítulo 4.

Finalmente, no capítulo 6 são apresentadas as considerações finais e perspectivas de trabalhos futuros

%==============================================================%

