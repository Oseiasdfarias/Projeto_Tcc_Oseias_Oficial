% -------------------------------------------------------------------------------------------------
% Informações de dados para CAPA e FOLHA DE ROSTO
% -------------------------------------------------------------------------------------------------

%TÍTULO DO TRABALHO DE CONCLUSÃO DE CURSO (EM CAIXA ALTA):
\titulo{\textbf{DESENVOLVIMENTO DE PROTÓTIPO E GÊMEO DIGITAL COMO FERRAMENTA PARA UM LABORATÓRIO VIRTUAL COM FOCO EM MODELAGEM E CONTROLE DE SISTEMAS DINÂMICOS}}

%DADOS DO AUTOR (EM CAIXA ALTA):
\autor{OSÉIAS DIAS DE FARIAS}
%\autorcitacao{LISBOA, Ricardo} % Apenas o último sobrenome em CAIXA ALTA

%DADOS DOS ORIENTADORES:
\orientador{Prof. Dr. Raphael Barros Teixeira}

%LOCAL E ANO (EM CAIXA ALTA):
\local{TUCURUÍ}
\data{2023}

%INSTITUIÇÕ DE ENSINO (EM CAIXA ALTA):
\instituicao{%
  UNIVERSIDADE FEDERAL DO PARÁ
  \par
  CAMPUS UNIVERSITÁRIO DE TUCURUÍ
  \par
  FACULDADE DE ENGENHARIA ELÉTRICA}

%\tipotrabalho{TD: 13/2022}

% Preambulo para as folhas de rosto e aprovação:
\preambulo{Trabalho de Conclusão de Curso, apresentado como requisito parcial para a obtenção de
grau de Bacharel em Engenharia Elétrica, pela Universidade Federal do Pará.}
% ---



%
%% NATUREZA DO TRABALHO-----------------------------------------
%\projeto{Trabalho de Conclusão de Curso}
%
%% TÍTULO ACADÊMICO---------------------------------------------
%\tituloAcademico{Bacharel}
%
%% ÁREA DE CONCENTRAÇÃO E LINHA DE PESQUISA--------------------
%% Se a natureza for Trabalho de Conclusão de Curso, deixe ambos os campos vazios
%% Se for programa de Pós-graduação, indique a área de concentração e a linha de pesquisa
%\areaconcentracao{}
%\linhapesquisa{}
%
%% DADOS DA INSTITUIÇÃO-----------------------------------------
%\instituicao{Universidade Federal do Pará}
%\departamento{CAMPUS UNIVERSITÁRIO DE TUCURUÍ}
%\programa{Faculdade de Engenharia Elétrica}
%\logoinstituicao{2.5cm}{figuras/naomexafig/logoufpa.png} %
%
%% DADOS DOS ORIENTADORES----------------------------------------
%\orientador{Prof. Dr. André Felipe S. da Cruz}
%%\orientador[Orientadora:]{Nome da orientadora}
%\instOrientador{}
