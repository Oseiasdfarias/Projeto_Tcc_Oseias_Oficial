Para o desenvolvimento dos softwares destinados à visualização de dados e ao simulador, a linguagem de programação escolhida foi o Python. Essa seleção se deu devido à natureza versátil da linguagem, que permite um desenvolvimento ágil de softwares para uma ampla gama de finalidades. Em paralelo, para a programação do microcontrolador, optou-se pelo uso das linguagens C/C++. Essa escolha se fundamenta na ampla adoção dessas linguagens na programação de sistemas embarcados, garantindo um ambiente propício para a eficaz implementação no microcontrolador.

A linguagem Python é amplamente reconhecida como uma linguagem de programação de alta qualidade, interpretada e de propósito geral. Ela desfruta de uma imensa popularidade em todo o mundo e é empregada em diversas aplicações, abrangendo campos como computação científica, ciência de dados, engenharia de software e inteligência artificial. Python se destaca por ser uma linguagem de programação de fácil aprendizado e utilização, mesmo para indivíduos com pouca experiência em codificação. Sua sintaxe clara e concisa contribui para a legibilidade e manutenibilidade do código.

Por essas razões, Python emerge como a escolha ideal para conduzir trabalhos de pesquisa e desenvolvimento. É uma ferramenta incrivelmente versátil e poderosa, capaz de lidar com uma extensa variedade de tarefas, desde a análise de dados até a criação de aplicações complexas.


Já C e C++ são linguagens de programação que se concentram na eficiência e na portabilidade. Elas são amplamente utilizadas no desenvolvimento de sistemas operacionais, drivers de dispositivos, compiladores e outros softwares críticos. C foi criada em 1972 por Dennis Ritchie para o desenvolvimento do sistema operacional Unix. C++ foi criada em 1983 por Bjarne Stroustrup como uma extensão de C para adicionar suporte à programação orientada a objetos.

C e C++ são linguagens de programação poderosas e flexíveis, mas também podem ser complexas e difíceis de aprender. Elas são recomendadas para desenvolvedores que precisam de um alto nível de controle sobre o desempenho e a portabilidade de seu código.