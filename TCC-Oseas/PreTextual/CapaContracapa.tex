	%% Capa
	\thispagestyle{empty}
	\begin{center}

        \includegraphics[scale=0.5]{Figuras/logoufpa.png}
		
			UNIVERSIDADE FEDERAL DO PARÁ\\
			CAMPUS UNIVERSITÁRIO DE TUCURUÍ\\
			FACULDADE DE ENGENHARIA ELÉTRICA
			
			\vfill 
			
			\textbf{AEROPÊNDULO, PROTOTIPAGEM E SIMULADOR GRÁFICO COMO FERRAMENTA PARA ESTUDO DE TÉCNICAS DE CONTROLE E IDENTIFICAÇÃO DE SISTEMAS}
			
			\vfill
			
			\textbf{OSÉIAS DIAS DE FARIAS}
			
			\vfill \vfill
			
			Tucuruí-PA\\			
			2023
			
	\end{center}

	%% Página em branco:
	\newpage
	\thispagestyle{empty}
		%% Contracapa: ---------------------------------------------------
		\newpage
		\pagenumbering{roman}
		\setcounter{page}{2}
		\newpage
		\thispagestyle{empty}
		\begin{center}
  
            \includegraphics[scale=0.5]{Figuras/logoufpa.png}
            
			UNIVERSIDADE FEDERAL DO PARÁ\\
			CAMPUS UNIVERSITÁRIO DE TUCURUÍ\\
			FACULDADE DE ENGENHARIA ELÉTRICA
			
			\vspace{7mm}
			
			\vfill
			
			\textbf{OSÉIAS DIAS DE FARIAS}
			
			\vfill
			
			\vspace{10mm}
			
			\textbf{AEROPÊNDULO, PROTOTIPAGEM E SIMULADOR GRÁFICO COMO FERRAMENTA PARA ESTUDO DE TÉCNICAS DE CONTROLE E IDENTIFICAÇÃO DE SISTEMAS}
						
			\vfill\vfill
			
			\begin{flushright}				
				{\setlength{\fboxsep}{0pt}}
				
				\fbox{\begin{minipage}{7.1cm}\footnotesize
						Trabalho de conclusão de curso apresentado ao colegiado da Faculdade de Engenharia Elétrica, do Campus Universitário de Tucuruí, da Universidade Federal do Pará, como  requisito necessário para obtenção do título de Bacharel em Engenharia Elétrica.
				\end{minipage}}\\
				\vspace{10pt}
				\begin{flushleft}
				\hspace{8.9cm}	{\footnotesize \textbf{Orientador}: Prof. Dr. Raphael Barros Teixeira} \\
				\end{flushleft}
			\end{flushright}
			\vfill\vfill
			Tucuruí-PA\\			
			2023
		\end{center}